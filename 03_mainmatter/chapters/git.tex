\chapter{Git + GitHub}

\section{Git}

\lipsum[1]
\begin{wrapfigure}{r}{0.25\paperwidth}
	\includegraphics[width=0.24\paperwidth]{./img/git}
\end{wrapfigure}
\lipsum[1]

\begin{description}
	\item[Repositorio] Es una base de datos que contiene toda la información acerca de las versiones de los archivos en un proyecto, ubicado en una carpeta oculta lstinline|.git|.
	\item[Confirmación] Instantánea global de todos los archivos del proyecto con descripción de los cambios con respecto a la versión anterior.
	\item[Rama] Secuencia de commits que describe una rama (línea) de desarrollo. Un repositorio puede contener más de uno. La rama estándar se llama lstinline|master|.
	\item[Tag] Nombre persistente (identificador) para un commit, por ejemplo, cuando hace un lanzamiento público.
\end{description}

Antes de empezar a usar git, debemos tener instalado el sistema de control de versiones.
\begin{lstlisting}
four-leaves@me:~$ sudo apt update
four-leaves@me:~$ sudo apt install git-core
\end{lstlisting}
Muy bien, ahora dos cosas deben ser configuradas:
\begin{lstlisting}
four-leaves@me:~$ git config --global user.name "testuser"
four-leaves@me:~$ git config --global user.email "testuser@example.com"
\end{lstlisting}
%TODO: Nota: bashrc, bashprofile

Cree un repositorio git ingresando lo siguiente en el directorio superior del proyecto:
\begin{lstlisting}
four-leaves@me:~$ git init miProyecto
Initialized empty Git repository in ~/miProyecto.git/
\end{lstlisting}
Esto ha creado un repositorio vacío, todavía no hay confirmaciones:
\begin{lstlisting}
four-leaves@me:~$ git status
On branch master

No commits yet

nothing to commit (create/copy files and use "git add" to track)
\end{lstlisting}

\begin{lstlisting}
.git
├── branches
├── config
├── description
├── HEAD
├── hooks
│   ├── applypatch-msg.sample
│   ├── commit-msg.sample
│   ├── fsmonitor-watchman.sample
│   ├── post-update.sample
│   ├── pre-applypatch.sample
│   ├── pre-commit.sample
│   ├── pre-merge-commit.sample
│   ├── prepare-commit-msg.sample
│   ├── pre-push.sample
│   ├── pre-rebase.sample
│   ├── pre-receive.sample
│   └── update.sample
├── info
│   └── exclude
├── objects
│   ├── info
│   └── pack
└── refs
├── heads
└── tags

9 directories, 16 files
\end{lstlisting}

\subsection{Confirmaciones}
Una confirmación contiene
\begin{itemize}
	\item una instantánea de todos los archivos.
	\item la fecha de creación.
	\item el nombre y contenido con respecto al autor de los cambios.
	\item una lista de cambios.
	\item una lista de referencias a confirmaciones principales.
\end{itemize}
Las confirmaciones son identificadas usando un hash (suma de verificación) de su contenido, por ejemplo lstinline|5cd006a1c044b786ea8196636cd0b62e296dbbe5|.

Los hash de confirmación pueden abreviarse siempre que la abreviatura sea única.

Las confirmaciones no pueden cambiar: contenido diferente $\Rightarrow$ valor hash diferente.

\subsection{Ramas}

\begin{itemize}
	\item Una rama es una línea de desarrollo dentro de un repositorio.
	\item Los repositorios pueden contener un número arbitrario de ramas.
	\item lstinline|git status| puede decirle en qué rama se encuentra.
	\item Una rama siempre apunta a un lstinline|commit|.
	\item La creación de una nueva confirmación guarda la confirmación actual como principal y la rama luego apunta la nueva confirmación.
\end{itemize}

\subsection{Varios repositorios}

\begin{itemize}
	\item git puede sincronizar cambios entre repositorios.
	\item Los repositorios adicionales se llaman remotos.
	\item Las ramas de un repositorio remoto tienen el nombre del repositorio como prefijo.
	\item Puede clonar un repositorio remoto para obtener su contenido.
	\item No se pueden ver ramas de repositorios remotos directamente.
	\item En la primera revisión, git crea una rama de seguimiento.
	\item Los nuevos cambios se pueden descargar y combinar usando lstinline|git pull|.
\end{itemize}

\begin{table}[ht!]
	\caption{Comandos principales de alto nivel de git (fuente: \url{git-scm.com/docs})}
	\centering
	\begin{tabular}{@{} ll @{}}
	\toprule
	Comando & Breve descripción
	\tabularnewline
	\midrule
	\lstinline|init| & Crea un repositorio de Git vacío o reinicialice uno existente.
	\tabularnewline
	\lstinline|clone| & Clona un repositorio en un nuevo directorio.
	\tabularnewline
	\lstinline|add| & Agrega contenido del archivo al índice.
	\tabularnewline
	\lstinline|commit| & Registra los cambios en el repositorio.
	\tabularnewline
	\lstinline|log| & Muestra registros de confirmación.
	\tabularnewline
	\lstinline|diff| & Muestra cambios entre confirmaciones, confirmación y árbol de trabajo, etc.
	\tabularnewline
	\lstinline|branch| & Enumera, crea o elimina ramas.
	\tabularnewline
	\lstinline|checkout| & Cambia ramas o restaura archivos de árbol de trabajo.
	\tabularnewline
	\lstinline|merge|& Une dos o más historias de desarrollo juntas.
	\tabularnewline
	\lstinline|pull| & Obtenga e integre con otro repositorio o una sucursal local.
	\tabularnewline
	\lstinline|push| & Actualiza las referencias remotas junto con los objetos asociados.
	\tabularnewline
	\bottomrule
	\end{tabular}
\end{table}

\section{GitHub}
\lipsum[1]
\begin{wrapfigure}{l}{0.15\paperwidth}
	\includegraphics[width=0.14\paperwidth]{./img/github}
\end{wrapfigure}
\lipsum[1]

\section{GitLab}
\lipsum[1]
\begin{wrapfigure}{r}{0.15\paperwidth}
	\includesvg[width=0.14\paperwidth]{./img/gitlab}
\end{wrapfigure}
\lipsum[1]

\section{Automatización de compilación}

Organiza la construcción de su proyecto

Debe intentar ser eficiente y solo reconstruir lo que se requiere para mantener el ciclo corto de compilación-enlace-ejecución-depuración.

También puede intentar detectar propiedades del sistema de destino (a menudo el mismo que el host) y personalizar su programa en consecuencia.

También puede intentar proporcionar opciones de compilación para habilitar / deshabilitar opcionalmente las funciones de su proyecto.

También puede organizar pruebas de su proyecto.

Desea construir su proyecto lo más rápido posible.

Es posible que desee hacer que su proyecto sea portátil.

Escribir tu propio sistema de construcción es aburrido y tedioso.

Makefiles escritos a mano

Los desarrolladores especifican los objetivos y sus reglas de compilación.

Si escribió sus reglas correctamente, make asegura que solo se vuelvan a compilar los archivos de origen necesarios.

Perfecto para proyectos pequeños
\section{Lenguajes de marcado}

\subsection{Markdown}
\lipsum[1]
\begin{wrapfigure}{r}{0.2\paperwidth}
	\includegraphics[width=0.19\paperwidth]{./img/markup/markdown}
\end{wrapfigure}
\lipsum[1]

\subsection{ReStructuredText}
\lipsum[1]
\begin{wrapfigure}{l}{0.2\paperwidth}
	\includegraphics[width=0.19\paperwidth]{./img/markup/ReStructuredText}
\end{wrapfigure}
\lipsum[1]

\subsection{yaml}
\lipsum[1]
\begin{wrapfigure}{r}{0.2\paperwidth}
	\includegraphics[width=0.19\paperwidth]{./img/markup/yaml}
\end{wrapfigure}
\lipsum[1]

\subsection{orgmode}
Vi and emacs are the most popular editors of UNIX and Linux worlds.

Their flexibility and size of functionalities ...
\lipsum[1]
\begin{wrapfigure}{l}{0.2\paperwidth}
	\includesvg[width=0.19\paperwidth]{./img/markup/orgmode}
\end{wrapfigure}
\lipsum[1]